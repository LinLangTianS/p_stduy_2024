%导言区
\documentclass[a4paper,12pt]{ctexart} %A4纸,小四号字体
\usepackage{multirow}
\usepackage{amsmath, amsfonts, amssymb} 
\usepackage{xcolor}
\usepackage{mdframed} 
\usepackage{graphicx}   
\usepackage{booktabs,array}
\usepackage{listings} 
\usepackage{geometry}
\usepackage{indentfirst}
\usepackage{caption}
\usepackage{enumerate}
\usepackage{pifont}
\usepackage{CJK}
\usepackage{setspace}
\usepackage{url}
\usepackage{cleveref,cite,eqnarray}
\usepackage{subfigure}
\usepackage{booktabs}  
\usepackage{threeparttable}  %表格样式
\mdtheorem[style=theoremstyle]{defi}{$\blacklozenge$定义}
\renewcommand{\arraystretch}{1} %控制行高
\usepackage{amsmath} %花体2号
\usepackage{mathrsfs} %花体
\usepackage{amssymb}%花体3号
\usepackage{CJK}
\usepackage{setspace}
\usepackage{cleveref,cite,eqnarray}
\usepackage{subfigure}
\usepackage{subfigure}
\usepackage{float}
\pagestyle{plain}
\pagenumbering{Roman}
% 算法
\usepackage[noend]{algpseudocode}
\usepackage{algorithmicx,algorithm}
\floatname{algorithm}{方案}
\renewcommand{\algorithmicrequire}{\textbf{输入:}}
\renewcommand{\algorithmicensure}{\textbf{输出:}}

%设置代码边框颜色的
\usepackage{fontspec}
\newfontfamily\yaheiconsola{YaHei.Consolas.1.11b.ttf}
\setmonofont[
Contextuals={Alternate},
ItalicFont = Fira Code Retina Nerd Font Complete.otf     % to avoid font warning
]{YaHei.Consolas.1.11b.ttf}
\definecolor{codegreen}{rgb}{0,0.6,0}
\definecolor{NavyBlue}{rgb}{0.0, 0.0, 0.50}
\definecolor{PineGreen}{rgb}{0.0, 0.47, 0.44}
\lstset
{
	tabsize=4,
	captionpos=b,
	numbers=left,                    
	numbersep=1em,                  
	sensitive=true,
	showtabs=false, 
	frame=shadowbox,
	breaklines=true,
	keepspaces=true,                 
	showspaces=false,                
	showstringspaces=false,
	breakatwhitespace=false,         
	basicstyle=\yaheiconsola,
	keywordstyle=\color{NavyBlue},
	commentstyle=\color{codegreen},
	numberstyle=\color{gray},
	stringstyle=\color{PineGreen!90!black},
	rulesepcolor=\color{red!20!green!20!blue!20}
}
\geometry{a4paper,left=2.5cm,right=2.5cm,top=2.5cm,bottom=2.5cm} 
\pagestyle{plain}
\pagenumbering{Roman}
% 文章信息
\title{作业3——验收方案}
\author{舒双林 2024122141}
\date{\today}

%正文区
\begin{document}
	
\maketitle
\setcounter{page}{1}
\pagenumbering{arabic}
	
% \tableofcontents  % 目录
% \newpage

\section{待解决问题}
请建立数学模型,解决下述问题:

供应商声称一批零配件(零配件1或零配件2)的次品率不会超过某个标称值。企业准备采用抽样检测方法是否接受从供应商
购买的这批零配件,检测费用由企业自行承担,请为企业设计检测次数尽可能少的抽样检测方案。

如果标称值为$10\%$,根据你们的抽样检测方案,针对以下两种情形,分别给出具体结果:

(1)在$95\%$的信度下认定零配件次品率超过标称值,则拒收这批零配件;

(2)在$90\%$的信度下认为零配件次品率不超过标称值,则接收这批零配件。

\section{(N,n,c)验收方案}
\subsection{方案设计}

考虑一种(N,n,c)验收方案:在一批零配件中进行一次样本量为N的抽样检测,预先规定合格判定数为$c$。假设样本中的次品数量为n,认为
\begin{itemize}
	\item 当$n \leqslant c$时,认为这批零配件合格,接受该批产品;
	\item 当$n > c$时,认为这批零配件不合格,拒收该批产品;
\end{itemize}

不妨假设真实次品率$p_0 = 10\%$,定义$L(p)$为在次品率为$p$时接受这批零配件的概率,在零件总量十分庞大的情形下,将其视为二项分布,此时有
\begin{equation}
	L\left( p,c \right) =\sum_{k=0}^c{C_{N}^{k} p^k\left( 1-p \right) ^{N-k}}
\end{equation}

首先,根据二项分布,在样本量$N=30,c=3$的条件下,可以计算得出:
\begin{equation}
	L(p_0 ,c) =\sum_{k=0}^3{C_{30}^{k} p_0^k\left( 1-p_0 \right) ^{30-k}}=0.647439
\end{equation}

从而得到

\begin{align*}
	P\left( n \leqslant 3 \right) &=L(0.1,3)=0.647439
	\\
	P\left( n> 3  \right) &=1-P\left( n \leqslant 3 \right)=0.352561
\end{align*}
其中,n为这批零件检测到的次品个数。

其次,在大样本下,可将其视为正态分布,结合两种情形,分别构建置信度为$\alpha$的单侧置信区间,进行假设检验。


\subsection{仿真模拟}
\subsubsection{流程设定}
设定的判定流程如下所示:
\begin{center}
	\begin{minipage}{0.95\textwidth}
		\begin{algorithm}[H]%[!htp]
			\caption{(N,n,c)验收方案} %算法的名字
			\label{1}
			{\bf 输入:} 次品率$p$,样本数$N$,合格判定数$c$,模拟次数K\\
			{\bf 过程:} 
			\begin{algorithmic}[1]
				\State 令$K_{acc} = 0$为接受次数,$K_{rej} = 0$为拒绝次数
				\For{$i = 1$ to $K$} % For 语句,需要和EndFor对应
				\State $samples \sim B(N,p) $
				\State 令n等于二项分布中次品个数
				\If{$d \leqslant 1 $}
				\State $K_{acc} = K_{acc} + 1$
				\State 认为这批零配件合格;
				\Else
				\State $K_{rej} = K_{rej} + 1$
				\State 认为该批零配件不合格。
				\EndIf
				\State 	$i=i+1$
				\EndFor
				\State 	\textbf{end}
			\end{algorithmic}
			{\bf 输出:} %算法的结果输出
			$K_{acc}$,$K_{rej}$
		\end{algorithm}
	\end{minipage}
\end{center}

通过上述流程,对不同参数(p)下的$K_{acc}$,$K_{rej}$进行模拟,得到其分布情况,从而进行假设检验。

\subsubsection{问题(1)的模拟与检验}
取定次品率$p = 0.15$,应用python对方案进行模拟,得到如下结果:
\begin{table}[H]
    \centering
    \caption{模拟结果统计($4*1000$次,$p=0.15$)}
    \begin{tabular}{ccccc}
    \toprule[1.5pt]
        ~ & seed=0 & seed=1 & seed=2 & seed=3 \\ \midrule[0.75pt]
        接受次数 & 3276 & 3271 & 3318 & 3307 \\ 
        拒绝次数 & 6724 & 6729 & 6682 & 6693 \\ \bottomrule[1.5pt]
    \end{tabular}
\end{table}

紧接着进行假设检验。检验原假设与备择假设
$$H_0:p_0 \leqslant0.1 \quad vs \quad H_1:p_0>0.1$$

以接受率$L\left( p,c \right)$为检验统计量,取显著性水平$\alpha = 0.95$,拒绝域为$\left\{ L\leqslant L\left( P_0,c \right) +z_{\alpha}\sqrt{\frac{\hat{p}\left( 1-\hat{p} \right)}{N}} \right\} $


\subsubsection{问题(2)的模拟与检验}

\begin{appendix}

\section{附录}
\lstinputlisting[language=python]{123.py}
\end{appendix}
\end{document}
